\documentclass{article}


% if you need to pass options to natbib, use, e.g.:
%     \PassOptionsToPackage{numbers, compress}{natbib}
% before loading neurips_2024


% ready for submission
\usepackage{neurips_2024}


% to compile a preprint version, e.g., for submission to arXiv, add add the
% [preprint] option:
%     \usepackage[preprint]{neurips_2024}


% to compile a camera-ready version, add the [final] option, e.g.:
%     \usepackage[final]{neurips_2024}


% to avoid loading the natbib package, add option nonatbib:
%    \usepackage[nonatbib]{neurips_2024}


\usepackage[utf8]{inputenc} % allow utf-8 input
\usepackage[T1]{fontenc}    % use 8-bit T1 fonts
\usepackage{hyperref}       % hyperlinks
\usepackage{url}            % simple URL typesetting
\usepackage{booktabs}       % professional-quality tables
\usepackage{amsfonts}       % blackboard math symbols
\usepackage{nicefrac}       % compact symbols for 1/2, etc.
\usepackage{microtype}      % microtypography
\usepackage{xcolor}         % colors

% Packages added by me.
\usepackage{amsmath}
\usepackage{amssymb}
\usepackage{amsthm}
\usepackage{todonotes}


\title{Title}


% The \author macro works with any number of authors. There are two commands
% used to separate the names and addresses of multiple authors: \And and \AND.
%
% Using \And between authors leaves it to LaTeX to determine where to break the
% lines. Using \AND forces a line break at that point. So, if LaTeX puts 3 of 4
% authors names on the first line, and the last on the second line, try using
% \AND instead of \And before the third author name.


\author{%
  David S.~Hippocampus\thanks{Use footnote for providing further information
    about author (webpage, alternative address)---\emph{not} for acknowledging
    funding agencies.} \\
  Department of Computer Science\\
  Cranberry-Lemon University\\
  Pittsburgh, PA 15213 \\
  \texttt{hippo@cs.cranberry-lemon.edu} \\
  % examples of more authors
  % \And
  % Coauthor \\
  % Affiliation \\
  % Address \\
  % \texttt{email} \\
  % \AND
  % Coauthor \\
  % Affiliation \\
  % Address \\
  % \texttt{email} \\
  % \And
  % Coauthor \\
  % Affiliation \\
  % Address \\
  % \texttt{email} \\
  % \And
  % Coauthor \\
  % Affiliation \\
  % Address \\
  % \texttt{email} \\
}


\begin{document}


\maketitle


\begin{abstract}
	Causal discovery is a fundamental problem in causal inference, and
	numerous algorithms have been developed to address it. However, their
	adoption in applied disciplines  has been limited, possibly due to a
	lack of confidence in these algorithms stemming from them making
	obvious mistakes, and the difficulties in choosing the right algorithm
	for specific datasets. Consequently, in applied disciplines, these
	causal Directed Acyclic Graphs (DAGs) are predominantly constructed
	manually based on domain expertise. This makes it vital to test and
	potentially modify these models against data before using them for any
	analysis. Local testing based on Conditional Independence tests and
	their effect sizes have been commonly used for this purpose. However,
	one of the current limitations of this approach is that there are no
	effect sizes for mixed data. To address this, we introduce a novel
	effect size measure for mixed data based on canonical correlations.
	This effect size is a generalization of some of the effect sizes used
	in special cases to mixed variables. Moreover, to assist researchers in
	constructing DAGs manually, we use the proposed mixed data effect size
	measure to build a web-based tool that can help researchers in
	constructing, validating, and modifying their DAGs with the help of
	available data. The tool iteratively suggests modifications to DAGs
	based on correlations that are not explained/over-explained by the DAG,
	while the expert chooses the direction of the edge between those
	variables. Empirically, we show that taking a greedy approach on the
	tool's recommendations can perform comparably to causal discovery
	algorithms, provided that the expert correctly determines the direction
	of the edges in at least one out of three cases, with performance
	improving as accuracy increases.
\end{abstract}

\section{Introduction}
\textbf{Para 1: Many Causal discovery algorithms but commonly constructed manually.}

To perform any causal effect identification or estimation in Pearl's Directed
Acyclic Graphs (DAGs) based causal inference usually requires to come up with
the DAG structure. Learning these DAG structures from data is known as
\emph{causal discovery}. Plenty of algorithms have been developed for these in
the literature such as constraint-based methods such as PC, FCI, score based
methods such as Hill-Climb Search, GES, and constrained continuous optimization
based such as No TEARS.

\textbf{Para 2: Reasons behind this}

Even with the existence of all these algorithms for constructing DAGs from
data, in applied fields researchers prefer to construct DAGs by hand. We
hypothesize that the adaption of these algorithms in applied literature is
low due to the following reasons:
\begin{enumerate}
	\item Algorithms can make very obvious mistakes at times making it harder
		for the user to trust their outputs.
	\item Difficult to choose a single algorithm and their hyper-parameters.
\end{enumerate}

\textbf{Para 3: Importance of testing, include modification indices}

When these models are constructed manually, it is extremely important that they
are tested before they are used for making any inference. Local testing methods
have been used for this purpose, where the implied conditional independences of
the model are testing in the data using statistical tests. These tests can then
be used to modify our models.

\textbf{Para 4: Mixed data testing}

One of the limitations here is there are no effect size measures available for mixed data.

\textbf{Para 5: Contributions}
\begin{enumerate}
	\item An effect size measure for mixed data. 
	\item A web-based tool to help with both model testing and construction. Combining both domain knowledge and data.
\end{enumerate}


\section{Background}
\todo[inline]{Add notation used here}

\subsection{Continuous Variables}
Correlation coefficient and partial correlation coefficient.

\subsection{Discrete Variables}
Chi-Squared and Cram\'er's V

\subsection{Ordinal and Continuous Variables}
Polyserial Correlation

\subsection{Ordinal Variables}
Polychoric Correlation

\subsection{Mixed data}
Background on CI tests for mixed data; but no effect size measure.

\section{Effect size for mixed data}

\section{Web Tool}
As most of the DAGs in practice are built manually, we use our mixed data effect size measure to build a web-tool that aids researches in building DAGs.

\section{Empirical Analysis}
We performed empirical analysis to compare how well does creating a graph manually based on effect sizes and p-value compare to algorithmic approaches.
\subsection{Simulating an expert}
To simulate an expert building a model manually, we wrote a simple greedy simulation as shown in the Figure. The expert always starts with a graph with
no edges and chooses the pair of variables with the highest correlation between them. The expert based on some accuracy measure decides the direction
of the edge between these variables such that the new added edge does not create a cycle in the DAG:

\todo{Summarize this section and equaiton in an algorithm}
\begin{equation}
	\begin{split}
		x &= \textnormal{rand}([0, 1]) \\
		O(\alpha) &= \begin{cases} 
			M \rightarrow Y, & \textnormal{if  } x <= \alpha \\
			\textnormal{rand}(M \rightarrow Y, M \leftarrow Y, None) & \textnormal{otherwise} \\
			     \end{cases} \\
	\end{split}
\end{equation}

After adding each edge, we also check if the p-value of any existing edge shows independence. If that happens we remove that edge and blacklist that edge. 
Blacklisted edges are not suggested to the expert in future iterations. We repeat this till we have explained all the correlations in the dataset.

This is a possibility that this greedy expert gets stuck at incorrect graph structures as it does not do any backtracking to fix its earlier mistakes.


\subsection{Results}

\section{Conclusions}
\subsection{Using LLMs as experts}



\bibliographystyle{plain}
\bibliography{references.bib}

%%%%%%%%%%%%%%%%%%%%%%%%%%%%%%%%%%%%%%%%%%%%%%%%%%%%%%%%%%%%

\appendix

\section{Appendix / supplemental material}


Optionally include supplemental material (complete proofs, additional experiments and plots) in appendix.
All such materials \textbf{SHOULD be included in the main submission.}

%%%%%%%%%%%%%%%%%%%%%%%%%%%%%%%%%%%%%%%%%%%%%%%%%%%%%%%%%%%%

\include{checklist}

\end{document}
