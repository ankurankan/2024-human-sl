\documentclass{uai2025} % for initial submission
%\documentclass[accepted]{uai2025} % after acceptance, for a revised version; 
% also before submission to see how the non-anonymous paper would look like 
                        
%% There is a class option to choose the math font
% \documentclass[mathfont=ptmx]{uai2025} % ptmx math instead of Computer
                                         % Modern (has noticeable issues)
% \documentclass[mathfont=newtx]{uai2025} % newtx fonts (improves upon
                                          % ptmx; less tested, no support)
% NOTE: Only keep *one* line above as appropriate, as it will be replaced
%       automatically for papers to be published. Do not make any other
%       change above this note for an accepted version.

%% Choose your variant of English; be consistent
\usepackage[american]{babel}
% \usepackage[british]{babel}

%% Some suggested packages, as needed:
\usepackage{natbib} % has a nice set of citation styles and commands
    \bibliographystyle{plainnat}
    \renewcommand{\bibsection}{\subsubsection*{References}}
\usepackage{mathtools} % amsmath with fixes and additions
% \usepackage{siunitx} % for proper typesetting of numbers and units
\usepackage{booktabs} % commands to create good-looking tables
\usepackage{tikz} % nice language for creating drawings and diagrams

\usepackage{amsmath}
\usepackage{amssymb}
\usepackage{amsthm}
\usepackage{todonotes}
\usepackage{bm}
\usepackage{subcaption}
\usepackage[linesnumbered, ruled]{algorithm2e}

\def\ci{\perp\!\!\!\!\!\perp}

\newtheorem{definition}{Definition}
\newtheorem{proposition}{Proposition}
\newtheorem{theorem}{Theorem}

\title{Expert-In-The-Loop Causal Discovery: Iterative Model Refinement Using Expert Knowledge}

% The standard author block has changed for UAI 2025 to provide
% more space for long author lists and allow for complex affiliations
%
% All author information is authomatically removed by the class for the
% anonymous submission version of your paper, so you can already add your
% information below.
%
% Add authors
\author[1]{\href{mailto:<ankur.ankan@ru.nl>?Subject=Your UAI 2025 paper}{Ankur~Ankan}{}}
\author[1]{Johannes~Textor}

% Add affiliations after the authors
\affil[1]{%
    Institute for Computing and Information Sciences\\
    Radboud University\\
    Nijmegen, The Netherlands
}
\begin{document}

\maketitle

\begin{abstract}

Causal discovery has received significant attention in the Directed Acyclic
Graphs (DAGs) literature, leading to the development of numerous automated
algorithms for learning DAGs from data. However, their adoption in applied
domains remain limited, as researchers often prefer to construct DAGs manually
based on domain knowledge. This preference arises due to several practical
challenges with automated algorithms, such as their tendency to make obvious
errors and output Markov Equivalence Classes (MECs) rather than a DAG. To
address these challenges, we propose an iterative method that guides
researchers in manually constructing or improving the fit of an existing DAG.
Our approach combines implied Conditional Independence (CI) testing with a
measure of conditional association between variables to rank violations
of implied CIs in a given DAG. This ranking helps prioritize modifications to
the DAG to improve its fit to the data while utilizing expert knowledge to
decide the orientation of new edges. Empirical results show that this guided
manual model construction approach achieves performance comparable to automated
algorithms if we are able to correctly decide the orientation of new edges in
one out of three cases. Additionally, we demonstrate that, in the absence of an
expert, a Large Language Model (LLM) can be potentially used to determine edge
orientations. We provide the implementation of our approach in both a web tool
at: <redacted for review> and a Python package <redacted for review>.

\end{abstract}

\section{Introduction}
Understanding cause-and-effect relationships between variables is a fundamental
objective in many scientific fields. These relationships reveal the mechanisms
behind observed phenomena and guide effective interventions or policy
decisions. Causal discovery methods aim to discover such relationships
among random variables using observational data. Approaches to causal discovery
have been developed within both the Directed Acyclic Graphs (DAGs) and
Structural Equation Models (SEMs) frameworks, each adapting a different
approach.

In the DAG literature, the primary focus has been on developing automated
algorithms to learn causal structures from datasets. These efforts have led to
numerous causal discovery algorithms, such as constraint-based methods like PC
algorithm \citep{Spirtes2001} and Fast Causal Inference \citep{Spirtes2000}),
score-based methods such as Hill-Climb Search and Greedy Equivalence Search
\citep{Chickering2002}, and continuous optimization-based methods like NO TEARS
\citep{Zheng2018} and DAGMA \citep{Bello2022}. While DAG-based methods focus on
automated discovery, SEM-based methods emphasize expert driven model
specification. This includes tools to assist researchers in manually
constructing models, enabling them to incorporate their domain knowledge in the
model building process. Researchers typically begin with an initial model based
on their domain knowledge and then use these tools to guide modifications that
improve the model's fit to data. This process is commonly known as
Specification Search \citep{Long1983} and uses method such as modification
indices, and Wald-based tests \citep{Marcoulides2018}. 

Despite significant progress in automated causal discovery, their adoption in
applied research has been limited. Researchers often prefer to manually
construct DAGs based on their domain expertise \citep{Tennant2020,
Petersen2021}. We attribute this preference to several challenges with existing
causal discovery algorithms in practical settings:

\begin{enumerate}
	\item \textbf{Lack of Trust:} While most algorithms are asymptotically
		consistent, their behavior on finite samples is not well
		understood. Their output can vary significantly depending on
		the choice of algorithm and hyperparameters, making it
		difficult to assess reliability. Additionally, the absence of
		robust performance evaluation methods for any given dataset
		further reduce the confidence in their outputs. 
	\item \textbf{Outputs Markov Equivalence Class (MEC):} As multiple
		DAGs can be faithful to a given observational dataset, automated 
		algorithms can only recover the MECs. These MECs can contain a
		combination of directed and undirected edges. However, most
		methods for downstream tasks, such as identification or causal
		effect estimation, assume knowledge of a fully oriented DAG. 
\end{enumerate}

Figure~\ref{fig:intro} highlights some of these issues. 

When manually constructing models, it is important to test whether the
resulting DAG accurately represents the dataset. Based on this evaluation and
domain knowledge, we can make further modification to the model. One common
approach for assessing DAGs is to check whether the Conditional Independences
(CIs) implied by the DAG hold in the data \citep{Ankan2021}. If we find
violations to these CIs, we can use our domain knowledge to make appropriate
modifications to the DAG. However, this CI testing approach has a few
limitations:

\begin{enumerate}
\item As CIs are only implied by missing edges in the model, this approach does 
	not test whether the existing edges are correct.
\item Determining whether a CI holds in data is based on a p-value threshold. This can 
	be unreliable, for example, we get a significant p-value for even very weak 
	relationship if the sample size is high.
\item The number of implied CIs can be large making it difficult to manually
	check all of them. For example, a moderately sized alarm network
	\citep{Beinlich1989} with $ 37 $ nodes, implies $ 287 $ CIs.
\end{enumerate}

To tackle these limitations, we draw inspiration from modification indices in
specification search. Modification indices provide ranking for potential model
modification based on their impact on the model's fit, allowing us to
prioritize most meaningful modifications.  Similarly, we propose a method for
ranking CI test violations to help us focus on the most critical
inconsistencies in the model. To determine this ranking we utilize a measure of
conditional association between variables in combination with p-values from the
CI test. This approach addresses all three of the issues outlined above. The
measure of association allows us to evaluate the validity of existing edges. It
reduces our reliance on p-values alone for deciding whether the CI test holds.
The ranking helps us prioritize the most critical violations, making it feasible 
to focus on the areas that most improve the model's fit.

\begin{figure}[t!]
    \begin{subfigure}{0.5 \textwidth}
	\centering
    	\includegraphics[page=1]{figures_v2.pdf}
    	\caption{}
    \end{subfigure}
    \begin{subfigure}{0.5\textwidth}
	\centering
    	\includegraphics[page=2]{figures_v2.pdf}
    	\caption{}
    \end{subfigure}
    \begin{subfigure}{0.5\textwidth}
	\centering
    	\includegraphics[page=3]{figures_v2.pdf}
    	\caption{}
    \end{subfigure}

    \caption{A comparison of Markov Equivalence Class (MEC) learned using Adult
	    Income Dataset \citep{Becker1996} using different algorithms and
	    sample sizes. Edge colour represents the sample size used: Red
	    ($N=400$), and blue ($N=800$). (a) PC algorithm with a mutual
	    information based CI test, (b) PC algorithm with a residualization
	    based test \citep{Ankan2023}, (c) Hill Climb Search with Bayesian
	    Information Criterion (BIC) score. The learned model structure
	    varies significantly in each case.}
    \label{fig:intro}
\end{figure}


Our main contributions in this paper are as follows:

\begin{enumerate}
    \item We propose a novel measure of conditional association for mixed data
	    based on canonical correlations
	    (Section~\ref{sec:mixed_association}). This measure generalizes
	    several commonly used special case measures of association to mixed
	    data.
    \item Using this measure of conditional association, we develop a procedure
	    to rank violations of implied CIs. Using this ranking helps us
	    prioritize modifications to DAG to improve its fit while allowing
	    us to integrate expert knowledge when adding new edges (Section~\ref{sec:modification}).
    \item We provide a web-based interactive tool and a Python package to allow 
	    researchers to easily apply this method to their datasets (Section~\ref{sec:web}).
    \item Lastly, we compare the manual DAG construction method with automated
	    causal discovery algorithms (Section~\ref{sec:empirical}).
\end{enumerate}

% The rest of the paper is structured as follows. In
% Section~\ref{sec:background}, we give a background on the commonly used
% measures of association for various data types. In
% Section~\ref{sec:mixed_association}, we present our generalized measure of
% conditional association for mixed data. Section~\ref{sec:modification} provides
% details on the procedure for using the measure of association for constructing
% DAGs. Section~\ref{sec:web} presents our web-browser based tool and lastly in
% Section~\ref{sec:empirical}, we show some empirical results to compare this
% manual DAG construction approach to automated algorithms.

\section{Background and Related Work}
\label{sec:background}
\todo[inline]{Need to rewrite this}
We denote random variables with uppercase letters $ X $, and a set of random
variables as $ \bm{X} = \{X_1, \cdots, X_k\} $ with $ \rvert \bm{X} \rvert = k
$. A sample from the random variable $ X $ is denoted as $ x $ and from a set
of random variables $ \bm{X} $ is denoted as $ \bm{x} $. In this paper, we
consider random variables in the mixed data setting where each of the random
variables can be either of continuous, categorical, or ordinal unless
specified. We write the expectation of a variable $ X $ as $ \mathbb{E}[X] $,
conditional expectations as $ \mathbb{E}[X | \bm{Z}] $. We denote a graph $ G =
(V, E) $ which nodes $ V$ and edges $ E $. In this paper we focus on all
observed variables, and linear measure of association. 
where $ \mathrm{cov}(X, Y) $ is the covariance between $ X $ and $ Y $, $ \mathrm{corr}(X, Y) $ is the correlation between $ X $ and $ Y $, and $
\sigma_X $ and $ \sigma_Y $ are the standard deviations of $ X $ and $ Y $
respectively.  Covariance matrix using $ \Sigma $ and the entry corresponding to
covariance between $ X $ and $ Y $ in the matrix using $ \Sigma_{XY} $.

\subsection{Measures of Conditional Association}
Our approach for ranking CI tests is based on a measure of conditional
association -- also known as partial association -- between variables.
Specifically, we are interested in quantifying the association between
variables $ X $ and $ Y $ when conditioned on a set of variables $ \bm{Z} $
(which may be empty, i.e., $ \bm{Z} = \emptyset $). Various measures of
conditional association have been used for this purpose depending on the type
of $ X $, $ Y $, and $ \bm{Z} $. These measures are often based on the effect
size of the CI test $ X \ci Y \rvert \bm{Z} $. In this section, we give an
overview of some of the commonly used measures for different data types.

\paragraph{Both $ X $ and $ Y $ are continuous: }
When both $ X $ and $ Y $ are continuous, Pearson's correlation coefficient is
typically used. When $ \bm{Z} = \emptyset $, the correlation
coefficient is defined as:

\begin{equation}
	r_{X, Y} = \frac{\mathrm{cov}(X, Y)}{\sigma_X \sigma_Y}
\end{equation}

When $ \bm{Z} \neq \emptyset $, partial correlation coefficient can be used.
This is estimated by fitting two regression models $ E_X: X \sim \bm{Z} $ and $
E_Y: Y \sim \bm{Z} $, calculating the residuals $ R_X = X - E_X(\bm{Z}) $ and $
R_Y = Y - E_Y(\bm{Y}) $, and then computing the correlation between the
residuals:

\begin{equation}
	r_{X, Y; \bm{Z}} = r_{R_X, R_Y}
\end{equation}

\paragraph{All $ X $, $ Y $, and $ \bm{Z} $ are discrete: }

When all variables are discrete, Cram\'er's V can used as a measure of
association. When $ \bm{Z} = \emptyset $, Cram\'er's V is derived from the 
chi-squared test statistic:

\begin{equation}
	\mathbf{V}_{X, Y} = \sqrt{\frac{\chi^2 / n}{\mathrm{min}(k-1, r-1)}}
\end{equation}

where $ \chi^2 $ is the chi-squared test statistic for $ X $ and $ Y $, $ n $
is the sample size, and $ k $ and $ r $ are the number of categories in $ X $
and $ Y $ respectively. When $ \bm{Z} \neq \emptyset $, we start by splitting
the dataset by each unique combination of $ \bm{Z} $ into subsets,  $ \bm{Z} $,
$ D = \{ D_{\bm{Z} = \bm{Z}_1}, D_{\bm{Z} = \bm{Z}_2}, \cdots, D_{\bm{Z} =
\bm{Z}_k} \} $. Then we compute the Cram\'er's V for each of these smaller datasets
and combine them as follows:

\begin{equation}
	\mathbf{V}_{X, Y; \bm{Z}} = \sum_{i=1}^{k} \left[ \mathbf{V}_{X, Y} \right]_{D_{\bm{Z} = \bm{Z}_i}} 
\end{equation}

\paragraph{$ X $ is ordinal, and $ Y $ and $ Z $  are continuous or ordinal: }

Polyserial (when one is ordinal and the other is continuous) and Polychoric
(when both are ordinal) correlation have been used to estimate the covariance
matrix between them \citep{Poon1987}. Both methods make the assumption that the
observed ordinal variable is a result of thresholding a latent normally
distributed continuous variable. Under this assumption, the methods then try to
estimate the covariance matrix while maximizing the likelihood of the dataset.
Using the estimated covariance matrix, $ \Sigma $ we can the compute Pearson's
correlation coefficient. When $ \bm{Z} = \emptyset $,

\begin{equation}
	r_{X, Y} = \frac{\Sigma_{XY}}{\sqrt{\Sigma_{XX} \Sigma_{YY}}}
\end{equation}

When $ \bm{Z} \neq \emptyset $, 

\begin{equation}
	r_{X, Y; \bm{Z}} = - \frac{\Sigma^{-1}_{XY}}{\sqrt{\Sigma^{-1}_{XX} \Sigma^{-1}_{YY}}}
\end{equation}


\section{DAG Structure Learning using Ancestral Oracles}
\label{sec:modification}

Our approach to causal structure learning combines domain knowledge with data-driven
insights in a manner that is based on the following base assumptions: 
(1) Domain experts are good at determining causal directions between variables if 
there is a clear causal direction between them. (2) Domain experts are less good
at identifying cases where there is no causal relationship between the variables,
since \emph{potential} causal relationships can often be argued for anyway.
(3) Domain experts are even less good at distinguishing direct from indirect 
effects, since the question whether a direct effect exists between two variables
depends on all other variables present in the graph. 

We therefore consider the following two models of a domain experts: Given two variables
\emph{X} and \emph{Y} that are part of an unknown causal DAG $G$, an \emph{strong ancestral oracle} 
$\mathcal{A}_G$ is defined as:
$$\mathcal{A}_G(X,Y)=\begin{cases}
 X \to Y & \textrm{if } X \in \textrm{An}_G(Y) \\
 X \gets Y & \textrm{if } Y \in \textrm{An}_G(X) \\
 \textrm{None} & \textrm{otherwise} \\
\end{cases}$$

whereas a \emph{weak ancestral oracle} is defined as:
$$
\mathcal{A}_G(X,Y)=\begin{cases}
 X \to Y & \textrm{if } X \in \textrm{An}_G(Y) \\
 X \gets Y & \textrm{if } Y \in \textrm{An}_G(X) \\
 \textrm{rand}(X \gets Y, X \to Y) & \textrm{otherwise} \\
\end{cases}
$$

That is, our ancestral oracles never distinguish whether the relationships
between variables is direct or indirect; and the weak ancestral oracle is even
unable to detect cases where there is no cause-effect relationship between two
variables. Using such oracles for DAG construction therefore inherently risks
that superfluous edges are created. In our approach, such superfluous edges
will be detected and removed by utilizing the data. 

To model how our learning procedure interacts with the data, we assume that we
have access to a second oracle $\mathcal{D}_G$ than can answer d-separation
queries with respect to the unknown target graph:
$\mathcal{D}_G(X,Y,\mathbf{Z})=1$ iff $X \ci_G Y \mid \mathbf{Z}$. These are
the standard oracles considered in constraint-based structure learning
algorithms, such as the PC algorithm \cite{Spirtes2001}. Based on this oracle,
we now define two procedures that iteratively change a current DAG structure.

Our structure learning algorithm is based on two core procedures.
\textsc{Expand} (Algorithm~\ref{algo:expand}) uses conditional independence
information to search for missing edges in the graph and then uses domain
knowledge to orient them. \textsc{Prune} (Algorithm~\ref{algo:prune}) uses
conditional independence information to remove superfluous edges from the
graph.

\begin{algorithm}[h]
\DontPrintSemicolon
\SetAlgoLined
\SetKwFunction{Expand}{Expand}
\SetKw{KwGoTo}{go to}
\SetKwProg{Fn}{Function}{:}{}
\Fn{{\sc Expand}($V,E,\mathcal{D}_G,\mathcal{A}_G$,$B=\emptyset$,$k=\infty$)}{
    let $L \gets \{\}$\;
    \ForEach{$X, Y$ where $X \to Y \notin E$ and $Y \to X \notin E$ and 
	$\{X,Y\} \notin B$}{
        let $\mathbf{Z}$ be a set that $d$-separates $X$ and $Y$ in $(V,E)$\;
        \If{$\mathcal{D}_G(X, Y, \mathbf{Z}) = 0$}{
            $L \gets L \cup \mathcal{A}_G(X, Y)$ \;
        }
	\If{$|L|\geq k$}{\KwGoTo 12\;}
    }
    add all edges in $L$ to $E$\;
    \Return{$(V,E)$}\;
}
\caption{Adding edges based on data and domain knowledge}
\label{algo:expand}
\end{algorithm}

The function $\textsc{Expand}$ takes an initial list of edges and searches for 
any vertex pairs that are not connected but where the d-separation oracle 
indicates  that there is still a residual association not explained by the other variables.
An optional parameter $B$, the empty set by default, allows to specify a ``black list''
of vertex pairs that must not be connected, and another optional parameter $k$
specifies the maximum amount of edges to be added by this procedure.

\begin{proposition}
For a conditional independence oracle
 $\mathcal{D}_G$ and a strong ancestral oracle $\mathcal{A}_G$, 
$\textsc{Expand}(V,\emptyset,\mathcal{D}_G,\mathcal{A}_G)=G^+$.
\end{proposition}

\begin{proposition}
For a conditional independence oracle
 $\mathcal{D}_G$ and a weak ancestral oracle $\mathcal{A}_G$, 
$\textsc{Expand}(V,\emptyset,\mathcal{D}_G,\mathcal{A}_G)\supseteq G^+$.
\end{proposition}


\begin{algorithm}[h]
\DontPrintSemicolon
\SetAlgoLined

\SetKwProg{Fn}{Function}{:}{}
\Fn{{\sc Prune}($V$,$E$,$\mathcal{D}_G$)}{
    let $L \gets \{\}$\;
    \ForEach{$X \to Y \in E$}{
        remove $X \to Y$ from $E$\;
        let $\mathbf{Z}$ be a set that $d$-separates $X$ and $Y$ in $(V,E)$\;
        \If{$\mathcal{D}_G(X, Y, \mathbf{Z}) = 1$}{
            $L \gets L \cup  \{X \to Y\}$ \;
        }
    }
    remove all edges in $L$ from $G$\;
    \Return{$(V,E)$}\;
}
\caption{Pruning superfluous edges}
\label{algo:prune}
\end{algorithm}




\begin{proposition}
Consider two DAGs $G=(V,E)$ and $G'=(V,E')$ where $E \subseteq E'$. 
Then $\textsc{Prune}(G',\mathcal{D}_G)=G$.
\label{prop:prune}
\end{proposition}

By combining the previous three propositions, we immediately obtain the following:

\begin{theorem}
Given a d-separation oracle $\mathcal{D}_G$ and a strong or weak 
ancestral oracle $\mathcal{A}_G$, 
$\textsc{Prune}(\textsc{Expand}((V,\emptyset),\mathcal{D}_G,\mathcal{A}_G),\mathcal{D}_G)=G.$
\end{theorem}

Since we require expert knowledge only in the $\textsc{Expand}$ operation, we may try
to be more economical by asking fewer questions at a time and interleaving expansion and
pruning steps. This leads us to the following, more incremental DAG construction algorithm.

\begin{algorithm}[h]
\DontPrintSemicolon
\SetAlgoLined
\SetKwProg{Fn}{Function}{:}{}

\Fn{{\sc ExpertInLoop}($V,\mathcal{D}_G,\mathcal{A}_G$)}{
$E_p \gets \emptyset$ \tcc{Current edges} 
$B \gets \emptyset$ \tcc{Edges that were pruned} 
\Repeat{ $E=E_p$ }{
	$E \gets E_p$ \;
	$(V,E) \gets \textsc{Expand}(V,E,\mathcal{D}_G,\mathcal{A}_G,B,1)$ \;
	$(V,E_p) \gets \textsc{Prune}(V,E,\mathcal{D}_G)$ \;
	$B \gets B \cup \{ E_p \setminus E \} $
}
\Return{(V,E)}
}

\caption{Iterative structure learning with expert in the loop}
\label{algo:expert}
\end{algorithm} 

\begin{theorem}
Let $G=(V,^*E)$ be a DAG, $\mathcal{D}_G$ a d-separation oracle for $G$, and 
$\mathcal{A}_G$ a strong or weak ancestral oracle for $G$. Then 
$\textsc{ExpertInLoop}(V,\mathcal{D}_G,\mathcal{A}_G)=G$.
\end{theorem}

\begin{proof}
The loop in Algorithm~\ref{algo:expert} terminates if and only if $E=E^*$. If the loop does not 
terminate, then a new edge has been added in line 6, or one or more edges were pruned in line 7.
Every edge can be added at 
most once and pruned at most once. Therefore, the algorithm always terminates
after at most $|V|(|V|-1)+1$ iterations of the loop. For every edge $e=X\to Y $ in the skeleton of $G$, $\mathcal{D}_G(X,Y,\mathbf{Z})=0$ irrespective of $\mathbf{Z}$. Therefore, $e$ must be added to $E$ in some iteration, and can never be pruned afterwards. Therefore, after some iteration, $(V,E)$ must be a supergraph of $G$ after executing line 6, and this will be pruned to the real graph $G$ in line 7 (Proposition~\ref{prop:prune}). In the next iteration, no further changes are made, and the loop terminates.
\end{proof} 


\begin{figure}[t!]
	\begin{subfigure}{0.125 \textwidth}
		\centering
		\includegraphics[page=1]{example.pdf}
		\caption{True DAG}
	\end{subfigure}%
	\begin{subfigure}{0.125 \textwidth}
		\centering
		\includegraphics[page=2]{example.pdf}
		\caption{No edges.}
	\end{subfigure}%
	\begin{subfigure}{0.125 \textwidth}
		\centering
		\includegraphics[page=3]{example.pdf}
		\caption{$ X_1 \rightarrow X_4 $}
	\end{subfigure}%
	\begin{subfigure}{0.125 \textwidth}
		\centering
		\includegraphics[page=4]{example.pdf}
		\caption{$ X_1 \rightarrow X_2 $}
	\end{subfigure}
	\begin{subfigure}{0.125 \textwidth}
		\centering
		\includegraphics[page=5]{example.pdf}
		\caption{$ X_1 \rightarrow X_3 $}
	\end{subfigure}%
	\begin{subfigure}{0.125 \textwidth}
		\centering
		\includegraphics[page=6]{example.pdf}
		\caption{$ X_2 \rightarrow X_4 $}
	\end{subfigure}%
	\begin{subfigure}{0.250 \textwidth}
		\centering
		\includegraphics[page=7]{example.pdf}
		\caption{$ X_3 \rightarrow X_4 $. $ X_1 \not \rightarrow X_4 $.}
	\end{subfigure}
	\caption{An example showing each iteration of the \emph{ExpertInLoop} algorithm. The dashed edges show all the potential edges that \emph{Expand} step could add in each iteration.}
\end{figure}



% \begin{figure}
% 	\centering
% 	\begin{subfigure}{0.5\textwidth}
% 		\includegraphics[scale=0.25]{../../presentations/2024_05_das/2.png}
% 	\end{subfigure}%
% 	\begin{subfigure}{0.5\textwidth}
% 		\includegraphics[scale=0.25]{../../presentations/2024_05_das/5.png}
% 	\end{subfigure}
% 	\caption{Screenshots of the web-tool. \todo[inline]{Insert screenshots of the web-tool}}
% \end{figure}

\section{Empirical Analysis}
\label{sec:empirical}

\begin{figure}[t!]
	\centering
	\begin{subfigure}{0.5\textwidth}
		\centering
		\includegraphics{../code/fig3/shd_ribbon.pdf}
		\caption{}
	\end{subfigure}
	\begin{subfigure}{0.5\textwidth}
		\centering
		\includegraphics{../code/fig3/sid_ribbon.pdf}
		\caption{}
	\end{subfigure}
	\caption{Comparison of PC, Hill-Climb Search, and GES algorithms against
		\emph{ExpertInLoop} algorithm. As automated algorithms only
		recover the CPDAG, we use the best and worst scoring
		orientation of the CPDAG to get the range. We test
		\emph{ExpertInLoop} with varying values of expert accuracy, $ \alpha = \{0.1, 0.3, 0.5, 0.7, 0.9\} $. The corresponding
		$\alpha_{\textrm{eff}} $ is shown in the plot.}
	\label{fig:shd_sid}
\end{figure}

\begin{figure}[t!]
	\begin{subfigure}{0.25\textwidth}
		\centering
		\includegraphics{../code/fig4/unexplained_effect.pdf}
		\caption{}
	\end{subfigure}%
	\begin{subfigure}{0.25\textwidth}
		\centering
		\includegraphics{../code/fig4/ll.pdf}
		\caption{}
	\end{subfigure}
	\caption{Plots showing improvements in fit of the model over iterations
	on the Adult income dataset. For the expert simulation an LLM is used
	for the edge orientations. (a) Shows using total unexplained effect that is the
	sum of effect size between variables that don't have an edge between them. (b)
	Shows the log-likelihood of the overall model.
	}
	\label{fig:unexplained_ll}
\end{figure}

The \emph{ExpertInLoop} algorithm requires two oracles, one that can answer
d-separation queries and an ancestral oracle that can give the orientation or
non-existence of edges between pairs of variables. For answering the
d-separation queries, we use a CI test, the partial correlation test with a
p-value threshold of $ 0.05 $. Secondly, to tell the direction between edges or
non-existence of edges, we simulate an expert with accuracy $ \alpha $ as
follows:

\begin{equation}
	\begin{split}
		x &= \textrm{rand}([0, 1]) \\
		E(\alpha) &= \begin{cases} 
			\mathrm{TrueDir}(X, Y),  & \textrm{if  } x <= \alpha \\
			\textrm{rand}(X \rightarrow Y, Y \leftarrow X, \textrm{None}) & \textrm{otherwise} \\
				\end{cases} \\
	\end{split}
\end{equation}

where given the true DAG, $ G $:

\begin{equation}
	\begin{split}
	\mathrm{TrueDir}(X, Y) &= \begin{cases}
					X \rightarrow Y, & \textrm{if } X \rightarrow Y \in G \\
					Y \rightarrow X, & \textrm{if } Y \rightarrow X \in G \\
					\textrm{None}, & \textrm{otherwise }
				  \end{cases}
	\end{split}
\end{equation}

An important point to note here is that the effective accuracy of $ E $ is
higher than $ \alpha $ as even when giving a random answer, there is a $ 1 $ in
$ 3 $ chance that the answer is correct. The effective accuracy is:

\begin{equation}
	\alpha_{\mathrm{eff}} = \alpha + (1 - \alpha) / 3
\end{equation}


We analyzed the performance of \emph{ExpertInLoop} algorithm by comparing it
with three other automated algorithms: PC, Hill-Climb Search, and Greedy
Equivalence Search (GES). For the analysis, we simulated $ 500 $ samples from
known DAGs and use the algorithms to recover the original DAG. We start by
generating a random DAG on $10$ nodes and use linear models with random effects
to generate the data. We vary the edge probability of the DAG to generate DAGs
with varying density. We repeat the data generation $30$ times for each density
value for the DAG. We compare the learned DAG to the original DAG using two
metrics: Structural Hamming Distance (SHD) and Structural Intervention Distance
(SID) \citep{Peters2015}.

\todo[inline]{Add equation for data generation method once the plot is finalized}

As the automated algorithms can only recover the CPDAG from data, and our
method recovers the DAG. To compare them fairly, we computed the SHD and SID
for all possible orientations of the CPDAG to show the best and worst case
scenarios. The results are shown in Figure~\ref{shd_sid}. For SHD, the
performance of \emph{ExpertInLoop} is comparable to the automated algorithms
for $ \alpha = 0.3 (\alpha_{\textrm{eff}} = 0.53 $, and performs better for
higher $ \alpha $ values. For SID, the \emph{ExpertInLoop} performs better in
denser DAGs.

% \begin{verbatim}
% 	Input: Dataset $ D $ on variables $ \bm{X} $
% 	Output: Learned DAG G.
% 	
% 	function compute_effects(G):
% 		pvalues <- dict()
% 		effects <- dict()
% 		for X_i, X_j in X:
% 			if edge between X_i, X_j:
% 				p_values[(X_i, X_j)] <- pvalue(X_i, X_j, pa_{\bar{G}}(X_i) \cup pa_{\bar{G}}(X_j))
% 				effects[(X_i, X_j)] <- \phi(X_i, X_j, pa_{\bar{G}}(X_i) \cup pa_{\bar{G}}(X_j))
% 			else:
% 				p_values[(X_i, X_j)] <- pvalue(X_i, X_j, pa_{\bar{G}}(X_i) \cup pa_{\bar{G}}(X_j))
% 				effects[(X_i, X_j)] <- \phi(X_i, X_j, pa_{\bar{G}}(X_i) \cup pa_{\bar{G}}(X_j))
% 		return p_values, effects
% 	
% 	$ G $ <- Empty Graph
% 	p_values, effects <- compute_effects(G)
% 	while (p_values > p_thres, 
% \end{verbatim}
% 
% \begin{algorithm}
% 	\KwData{Data set $ D $ on variables $ \bm{X} $, pthres, effectthres}
% 	\KwResult{The learned DAG: $ G $}
% 	\BlankLine
% 	$ G \leftarrow $ Empty Graph
% \end{algorithm}

\subsection{Connection to Score Based Methods}

Our procedure closely resembles certain score-based automated causal discovery
methods, such as Greedy Equivalence Search (GES), where the algorithm
iteratively adds or removes edges that improve the scoring metric the most.
Similarly, a greedy version of our approach would involve adding an edge
between the pair of variables with the highest unexplained correlation. While
both methods aim to find modifications that most improve the model at each
step, a key distinction lies in the nature of the evaluation criteria: unlike
standard scoring metrics, our measure of association is not decomposable. A key
property of scoring metrics is that they should be decomposible, i.e., they can
be expressed as the sum of scores of nodes given its parents. This means adding
or removing an edge affects only a localized part of the model. In contrast,
our measure of association is global, meaning that modifying an edge in one
part of the model can influence association values elsewhere.

Another major difference is in the interpretability of the evaluation metric.
Most scoring metrics are based on log-likelihood with a penalty for model
complexity. These score metrics allow for relative comparisons between models
but do not provide an absolute measure of model fit. That is, they indicate
which model is better for a given dataset but do not reveal how well the model
explains the data in an absolute sense. In contrast, summing our measure of
association across all variables provides a direct fitness measure—this sum
approaches zero when the model perfectly accounts for all observed correlations
in the data. This property enables an absolute assessment of model quality,
rather than just a relative comparison between models. Figure~\ref{fig:unexplained_ll}


\subsection{Using LLMs as Experts}
Recently, there has been a lot of work towards using LLMs for causal discovery,
with tasks ranging from determining pairwise edge orientation
\citep{Kiciman2023, Jin2024} to full causal structure learning \citep{Naik2023,
Vashishtha2023} and counterfactual reasoning tasks\citep{Kiciman2023} (see
\citet{Liu2024} for an overview).

Since our approach utilizes expert knowledge to determine edge orientations, we
investigated the potential of using an LLM for this task. We applied our causal
discovery procedure with using Gemini 1.5 flash model as the expert. Our method
uses a greedy approach where the pair of variables with the highest unexplained
correlation is oriented first. The prompt used to determine the orientation is
provided in Appendix~\ref{section:llms}. Our LLM prompt utilizes variable
description to determine the edge orientation. We applied this approach to the
adult income dataset and the output is shown in Fig.~\ref{fig:adult_llm}. 

Unlike the other approaches to utilize LLM for doing full causal discovery, we 
used a combination of statistical tests and LLMs. This greatly reduces the
amount of information that we need from the LLM. Our approach is also able to 
ask exact questions about the variables unlike some of the other approaches.

\section{Web Tool}
\label{sec:web}
To enable users to apply this method to their own datasets, we developed an
interactive web-based tool (shown in Fig.~\ref{fig:web}) for constructing
models. Users begin by uploading their dataset, which initializes an empty DAG
with nodes corresponding to the dataset\'s variables. They can then specify a
p-value threshold and a measure of association threshold.

Using the specified thresholds, the tool visually highlights unexplained
correlations by displaying red edges between variables where correlation exists
in the data but the current DAG is not able to explain. The thickness of these
edges represents the strength of the correlation, helping users prioritize
which edges to add. Similarly, if an edge in the graph is detected to be
unnecessary, it is highlighted in black. Based on this information, users can
select to remove unnecessary edge or select a potential edge to add and specify
the orientation of the edge. The tool computes Shipley’s C \citep{Shipley2000}
value at each change to determine the overall fit of the model to the data.
Once satisfied with the constructed DAG, users can export the model for further
analysis.

\begin{figure}[t!]
	\centering
	\includegraphics[scale=0.4]{../code/plots/web_tool_full_new.png}
	\caption{A screenshot of the web tool for constructing the model. Users
		can upload their dataset after which the tool creates an empty
		graph and shows all pair of variables which are associated in
		the model using undirected red edges with the strength of
		association represented using edge width. Users can then
		iteratively add edges to the model (shown in green) while
		deciding the edge orientation based on domain knowledge.
		Unnecessary edges are shown in black.}
	\label{fig:web}
\end{figure}

\begin{figure}[t!]
	\centering
	\includegraphics[page=1]{fig5.pdf}
	\caption{DAG learned using the adult income dataset using an LLM (GEMINI-1.5-flash) as the expert. The p-value threshold used is $ 0.05 $ and the measure of association threshold is $ 0. 1 $}
	\label{fig:adult_llm}
\end{figure}

\section{Conclusions}

Points:
\begin{itemize}
	\item As researchers prefer to anyways build models by hand, this is like a companion tool to help them in deciding the method.
	\item Solves all the $ 3 $ issues outlined in the introduction.
	\item Compared to other expert knowledge specification methods, users don't need to guess what information then need to provide as exact questions are asked. Also enables us to ask an LLM for it.
\end{itemize}

Future work:
\begin{itemize}
	\item Add latent variables; extend to ADMGs.
	\item Can potentially be combined with pairwise orientation methods.
	\item Improvements in the web-interface.
\end{itemize}

\bibliography{references}

\newpage

\onecolumn
\title{Expert-In-The-Loop Causal Discovery: Iterative Model Refinement Using Expert Knowledge \\ (Supplementary Material)}
\maketitle
\appendix

\section{Prompts Used for LLM}

\begin{figure}[ht!]
	\centering
	\begin{Verbatim}
You are an expert in Social Science. Following are the descriptions of two variables:

<A>: {description of variable A}
<B>: {description of variable B}

Which of the following two options is the most likely causal direction between these 
variables:

1. <A> causes <B>
2. <B> causes <A>

Return a single letter answer between the choices above; Do not provide any reasoning 
in the answer; Do not add any text formatting to the answer.
	\end{Verbatim}
	\caption{Prompt used for the LLM. Here the variable descriptions are replaced with description provided in Fig.~\ref{fig:var_description}}
	\label{fig:prompt}
\end{figure}

\begin{figure}[ht!]
	\begin{Verbatim}
          Age: The age of a person
    Workclass: The workplace where the person is employed such as Private industry, 
     	       or self employed
    Education: The highest level of education the person has finished.
MaritalStatus: The marital status of the person
   Occupation: The kind of job the person does. For example, sales, craft repair, 
   		clerical.
 Relationship: The relationship status of the person.
         Race: The ethnicity of the person.
          Sex: The sex or gender of the person.
 HoursPerWeek: The number of hours per week the person works.
NativeCountry: The native country of the person.
       Income: The income i.e. amount of money the person makes.
	\end{Verbatim}
	\caption{Variable descriptions used for prompting the LLM}
	\label{fig:var_description}
\end{figure}

\end{document}
