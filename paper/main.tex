\documentclass{uai2025} % for initial submission
%\documentclass[accepted]{uai2025} % after acceptance, for a revised version; 
% also before submission to see how the non-anonymous paper would look like 
                        
%% There is a class option to choose the math font
% \documentclass[mathfont=ptmx]{uai2025} % ptmx math instead of Computer
                                         % Modern (has noticeable issues)
% \documentclass[mathfont=newtx]{uai2025} % newtx fonts (improves upon
                                          % ptmx; less tested, no support)
% NOTE: Only keep *one* line above as appropriate, as it will be replaced
%       automatically for papers to be published. Do not make any other
%       change above this note for an accepted version.

%% Choose your variant of English; be consistent
\usepackage[american]{babel}
% \usepackage[british]{babel}

%% Some suggested packages, as needed:
\usepackage{natbib} % has a nice set of citation styles and commands
    \bibliographystyle{plainnat}
    \renewcommand{\bibsection}{\subsubsection*{References}}
\usepackage{mathtools} % amsmath with fixes and additions
% \usepackage{siunitx} % for proper typesetting of numbers and units
\usepackage{booktabs} % commands to create good-looking tables
\usepackage{tikz} % nice language for creating drawings and diagrams

\usepackage{amsmath}
\usepackage{amssymb}
\usepackage{amsthm}
\usepackage{todonotes}
\usepackage{bm}
\usepackage{subcaption}
% \usepackage{fancyvrb}
\usepackage[ruled]{algorithm2e}

\def\ci{\perp\!\!\!\!\!\perp}

\newtheorem{definition}{Definition}
\newtheorem{proposition}{Proposition}
\newtheorem{theorem}{Theorem}

\title{Expert-In-The-Loop Causal Discovery: Iterative Model Refinement Using Expert Knowledge}

% The standard author block has changed for UAI 2025 to provide
% more space for long author lists and allow for complex affiliations
%
% All author information is authomatically removed by the class for the
% anonymous submission version of your paper, so you can already add your
% information below.
%
% Add authors
\author[1]{\href{mailto:<ankur.ankan@ru.nl>?Subject=Your UAI 2025 paper}{Ankur~Ankan}{}}
\author[1]{Johannes~Textor}

% Add affiliations after the authors
\affil[1]{%
    Institute for Computing and Information Sciences\\
    Radboud University\\
    Nijmegen, The Netherlands
}
\begin{document}

\maketitle

\begin{abstract}

Causal discovery has received significant attention in the Directed Acyclic
Graphs (DAGs) literature, leading to the development of numerous automated
algorithms for learning DAGs from data. However, their adoption in applied
domains remain limited, as researchers often prefer to construct DAGs manually
based on domain knowledge. This preference arises due to several practical
challenges with automated algorithms, such as their tendency to make obvious
errors and output Markov Equivalence Classes (MECs) rather than a DAG. To
address these challenges, we propose an iterative method that guides
researchers in manually constructing or improving the fit of an existing DAG.
Our approach combines implied Conditional Independence (CI) testing with a
measure of conditional association between variables to rank violations
of implied CIs in a given DAG. This ranking helps prioritize modifications to
the DAG to improve its fit to the data while utilizing expert knowledge to
decide the orientation of new edges. Empirical results show that this guided
manual model construction approach achieves performance comparable to automated
algorithms if we are able to correctly decide the orientation of new edges in
one out of three cases. Additionally, we demonstrate that, in the absence of an
expert, a Large Language Model (LLM) can be potentially used to determine edge
orientations. We provide the implementation of our approach in both a web tool
at: <redacted for review> and a Python package <redacted for review>.

\end{abstract}

\section{Introduction}
Understanding cause-and-effect relationships between variables is a fundamental
objective in many scientific fields. These relationships reveal the mechanisms
behind observed phenomena and guide effective interventions or policy
decisions. Causal discovery methods aim to discover such relationships
among random variables using observational data. Approaches to causal discovery
have been developed within both the Directed Acyclic Graphs (DAGs) and
Structural Equation Models (SEMs) frameworks, each adapting a different
approach.

In the DAG literature, the primary focus has been on developing automated
algorithms to learn causal structures from datasets. These efforts have led to
numerous causal discovery algorithms, such as constraint-based methods like PC
algorithm \citep{Spirtes2001} and Fast Causal Inference \citep{Spirtes2000}),
score-based methods such as Hill-Climb Search and Greedy Equivalence Search
\citep{Chickering2002}, and continuous optimization-based methods like NO TEARS
\citep{Zheng2018} and DAGMA \citep{Bello2022}. While DAG-based methods focus on
automated discovery, SEM-based methods emphasize expert driven model
specification. This includes tools to assist researchers in manually
constructing models, enabling them to incorporate their domain knowledge in the
model building process. Researchers typically begin with an initial model based
on their domain knowledge and then use these tools to guide modifications that
improve the model's fit to data. This process is commonly known as
Specification Search \citep{Long1983} and uses method such as modification
indices, and Wald-based tests \citep{Marcoulides2018}. 

Despite significant progress in automated causal discovery, their adoption in
applied research has been limited. Researchers often prefer to manually
construct DAGs based on their domain expertise \citep{Tennant2020,
Petersen2021}. We attribute this preference to several challenges with existing
causal discovery algorithms in practical settings:

\begin{enumerate}
	\item \textbf{Lack of Trust:} While most algorithms are asymptotically
		consistent, their behavior on finite samples is not well
		understood. Their output can vary significantly depending on
		the choice of algorithm and hyperparameters, making it
		difficult to assess reliability. Additionally, the absence of
		robust performance evaluation methods for any given dataset
		further reduce the confidence in their outputs. 
	\item \textbf{Outputs Markov Equivalence Class (MEC):} As multiple
		DAGs can be faithful to a given observational dataset, automated 
		algorithms can only recover the MECs. These MECs can contain a
		combination of directed and undirected edges. However, most
		methods for downstream tasks, such as identification or causal
		effect estimation, assume knowledge of a fully oriented DAG. 
\end{enumerate}

Figure~\ref{fig:intro} highlights some of these issues. 

When manually constructing models, it is important to test whether the
resulting DAG accurately represents the dataset. Based on this evaluation and
domain knowledge, we can make further modification to the model. One common
approach for assessing DAGs is to check whether the Conditional Independences
(CIs) implied by the DAG hold in the data \citep{Ankan2021}. If we find
violations to these CIs, we can use our domain knowledge to make appropriate
modifications to the DAG. However, this CI testing approach has a few
limitations:

\begin{enumerate}
\item As CIs are only implied by missing edges in the model, this approach does 
	not test whether the existing edges are correct.
\item Determining whether a CI holds in data is based on a p-value threshold. This can 
	be unreliable, for example, we get a significant p-value for even very weak 
	relationship if the sample size is high.
\item The number of implied CIs can be large making it difficult to manually
	check all of them. For example, a moderately sized alarm network
	\citep{Beinlich1989} with $ 37 $ nodes, implies $ 287 $ CIs.
\end{enumerate}

To tackle these limitations, we draw inspiration from modification indices in
specification search. Modification indices provide ranking for potential model
modification based on their impact on the model's fit, allowing us to
prioritize most meaningful modifications.  Similarly, we propose a method for
ranking CI test violations to help us focus on the most critical
inconsistencies in the model. To determine this ranking we utilize a measure of
conditional association between variables in combination with p-values from the
CI test. This approach addresses all three of the issues outlined above. The
measure of association allows us to evaluate the validity of existing edges. It
reduces our reliance on p-values alone for deciding whether the CI test holds.
The ranking helps us prioritize the most critical violations, making it feasible 
to focus on the areas that most improve the model's fit.

\begin{figure}[t!]
    \begin{subfigure}{0.5 \textwidth}
	\centering
    	\includegraphics[page=1]{figures_v2.pdf}
    	\caption{PC algorithm with mutual information test. Red edges: $ N=400 $, Blue edges: $ N=800 $}
    \end{subfigure}
    \begin{subfigure}{0.5\textwidth}
    	\includegraphics[page=2]{figures_v2.pdf}
    	\caption{PC algorithm with a residualization based test. Red edges $N=400$, Blue edges: $ N=800 $}
    \end{subfigure}
    \begin{subfigure}{0.5\textwidth}
    	\includegraphics[page=3]{figures_v2.pdf}
    	\caption{PC algorithm with Hill Climb Search with BIC score}
    \end{subfigure}

    \caption{A comparison of Markov Equivalence Class (MEC) learned using Adult
	     Income Dataset \citep{Becker1996} using (a \& b) PC algorithm
	     using a mutual information based test, (c \& d) Hill Climb Search
	     with Bayesian Information Criterion (BIC) score, (e \& f) PC
	     algorithm with a residualization based test \citep{Ankan2023}. The
	     learned model structure varies significantly depending on the
     	     algorithm, the CI test, and the sample size used.}
    \label{fig:intro}
\end{figure}


Our main contributions in this paper are as follows:

\begin{enumerate}
    \item We propose a novel measure of conditional association for mixed data
	    based on canonical correlations
	    (Section~\ref{sec:mixed_association}). This measure generalizes
	    several commonly used special case measures of association to mixed
	    data.
    \item Using this measure of conditional association, we develop a procedure
	    to rank violations of implied CIs. Using this ranking helps us
	    prioritize modifications to DAG to improve its fit while allowing
	    us to integrate expert knowledge when adding new edges (Section~\ref{sec:modification}).
    \item We provide a web-based interactive tool and a Python package to allow 
	    researchers to easily apply this method to their datasets (Section~\ref{sec:web}).
    \item Lastly, we compare the manual DAG construction method with automated
	    causal discovery algorithms (Section~\ref{sec:empirical}).
\end{enumerate}

% The rest of the paper is structured as follows. In
% Section~\ref{sec:background}, we give a background on the commonly used
% measures of association for various data types. In
% Section~\ref{sec:mixed_association}, we present our generalized measure of
% conditional association for mixed data. Section~\ref{sec:modification} provides
% details on the procedure for using the measure of association for constructing
% DAGs. Section~\ref{sec:web} presents our web-browser based tool and lastly in
% Section~\ref{sec:empirical}, we show some empirical results to compare this
% manual DAG construction approach to automated algorithms.

\section{Background and Related Work}
\label{sec:background}
\todo[inline]{Need to rewrite this}
We denote random variables with uppercase letters $ X $, and a set of random
variables as $ \bm{X} = \{X_1, \cdots, X_k\} $ with $ \rvert \bm{X} \rvert = k
$. A sample from the random variable $ X $ is denoted as $ x $ and from a set
of random variables $ \bm{X} $ is denoted as $ \bm{x} $. In this paper, we
consider random variables in the mixed data setting where each of the random
variables can be either of continuous, categorical, or ordinal unless
specified. We write the expectation of a variable $ X $ as $ \mathbb{E}[X] $,
conditional expectations as $ \mathbb{E}[X | \bm{Z}] $. We denote a graph $ G =
(V, E) $ which nodes $ V$ and edges $ E $. In this paper we focus on all
observed variables, and linear measure of association. 
where $ \mathrm{cov}(X, Y) $ is the covariance between $ X $ and $ Y $, $ \mathrm{corr}(X, Y) $ is the correlation between $ X $ and $ Y $, and $
\sigma_X $ and $ \sigma_Y $ are the standard deviations of $ X $ and $ Y $
respectively.  Covariance matrix using $ \Sigma $ and the entry corresponding to
covariance between $ X $ and $ Y $ in the matrix using $ \Sigma_{XY} $.

\subsection{Measures of Conditional Association}
Our approach for ranking CI tests is based on a measure of conditional
association -- also known as partial association -- between variables.
Specifically, we are interested in quantifying the association between
variables $ X $ and $ Y $ when conditioned on a set of variables $ \bm{Z} $
(which may be empty, i.e., $ \bm{Z} = \emptyset $). Various measures of
conditional association have been used for this purpose depending on the type
of $ X $, $ Y $, and $ \bm{Z} $. These measures are often based on the effect
size of the CI test $ X \ci Y \rvert \bm{Z} $. In this section, we give an
overview of some of the commonly used measures for different data types.

\subsubsection{Both $ X $ and $ Y $ are continuous}
When both $ X $ and $ Y $ are continuous, Pearson's correlation coefficient is
typically used. When $ \bm{Z} = \emptyset $, the correlation
coefficient is defined as:

\begin{equation}
	r_{X, Y} = \frac{\mathrm{cov}(X, Y)}{\sigma_X \sigma_Y}
\end{equation}

When $ \bm{Z} \neq \emptyset $, partial correlation coefficient can be used.
This is estimated by fitting two regression models $ E_X: X \sim \bm{Z} $ and $
E_Y: Y \sim \bm{Z} $, calculating the residuals $ R_X = X - E_X(\bm{Z}) $ and $
R_Y = Y - E_Y(\bm{Y}) $, and then computing the correlation between the
residuals:

\begin{equation}
	r_{X, Y; \bm{Z}} = r_{R_X, R_Y}
\end{equation}

\paragraph{All $ X $, $ Y $, and $ \bm{Z} $ are discrete}

When all variables are discrete, Cram\'er's V can used as a measure of
association. When $ \bm{Z} = \emptyset $, Cram\'er's V is derived from the 
chi-squared test statistic:

\begin{equation}
	\mathbf{V}_{X, Y} = \sqrt{\frac{\chi^2 / n}{\mathrm{min}(k-1, r-1)}}
\end{equation}

where $ \chi^2 $ is the chi-squared test statistic for $ X $ and $ Y $, $ n $
is the sample size, and $ k $ and $ r $ are the number of categories in $ X $
and $ Y $ respectively. When $ \bm{Z} \neq \emptyset $, we start by splitting
the dataset by each unique combination of $ \bm{Z} $ into subsets,  $ \bm{Z} $,
$ D = \{ D_{\bm{Z} = \bm{Z}_1}, D_{\bm{Z} = \bm{Z}_2}, \cdots, D_{\bm{Z} =
\bm{Z}_k} \} $. Then we compute the Cram\'er's V for each of these smaller datasets
and combine them as follows:

\begin{equation}
	\mathbf{V}_{X, Y; \bm{Z}} = \sum_{i=1}^{k} \left[ \mathbf{V}_{X, Y} \right]_{D_{\bm{Z} = \bm{Z}_i}} 
\end{equation}

\paragraph{$ X $ is ordinal, and $ Y $ and $ Z $  are continuous or ordinal}

Polyserial (when one is ordinal and the other is continuous) and Polychoric
(when both are ordinal) correlation have been used to estimate the covariance
matrix between them \citep{Poon1987}. Both methods make the assumption that the
observed ordinal variable is a result of thresholding a latent normally
distributed continuous variable. Under this assumption, the methods then try to
estimate the covariance matrix while maximizing the likelihood of the dataset.
Using the estimated covariance matrix, $ \Sigma $ we can the compute Pearson's
correlation coefficient. When $ \bm{Z} = \emptyset $,

\begin{equation}
	r_{X, Y} = \frac{\Sigma_{XY}}{\sqrt{\Sigma_{XX} \Sigma_{YY}}}
\end{equation}

When $ \bm{Z} \neq \emptyset $, 

\begin{equation}
	r_{X, Y; \bm{Z}} = - \frac{\Sigma^{-1}_{XY}}{\sqrt{\Sigma^{-1}_{XX} \Sigma^{-1}_{YY}}}
\end{equation}

\section{A Generalized Measure of Conditional Association}
\label{sec:mixed_association}

In the previous section, we discussed how different types of variables require
different measures of conditional association. However, there is no unified
measure that handles mixed data. In this section, we introduce a measure of conditional
association for mixed data by extending the concept of partial correlation
coefficient (commonly used for continuous variables) to mixed data. Similar to
partial correlation coefficient, our method integrates a mixed data
residualization method \citep{Ankan2023} with Pillai's Trace
\citep{Pillai1955}, a multivariate measure of association based on canonical
correlations.
 
Given a dataset $ D = (x, y, \bm{z}) $ on variables $ X $, $ Y $, and $ \bm{Z}
$, our goal is to estimate the conditional association $ \phi_{X, Y; \bm{Z}} $. 
In the first step, we compute the residuals $ R_X $ and $ R_Y $ for variables
$ X $ and $ Y $ respectively. Depending on the type of variable the residual
is computed as follows:

\begin{enumerate}
	\item \textbf{Continuous:} We train a model, $ E_X: x \sim
		\bm{z} $. The residuals are then computed by taking the difference
		between the true and the predicted values using $ E_X $. 
		$$ R_{x_i} = x_i - E_X(\bm{z}_i) $$
	\item \textbf{Ordinal:} We start by training a probability estimator, $
		p_X: x \sim \bm{z} $, and then use the estimated probabilities, 
		$ \hat{p}_X(x) $ to compute the residuals:
		$$ R_{x_i} = \hat{p}_X(X < x_i) - \hat{p}_X(X > x_i) $$
	\item \textbf{Categorical:} We again start by training a probability
		estimator $ p_X: x \sim \bm{z} $, and obtain probability
		estimates $ \hat{p}_X: p_X(\bm{z}) $. Next, we dummy encode the
		categorical variable, resulting in a binary vector and then
		compute the residuals as follows: 
		$$ R_{x_i} = x_i - \hat{p}_X(\bm{z}_i) $$
\end{enumerate}

We have the option here to choose the estimators based on the characteristics
such as distribution, type of relationship, and so on of our dataset.
Non-parametric ensemble estimators such as Random Forest and XGBoost are robust 
for diverse data types and complex relationships. For linear relationships,
simpler models such as linear regression and its variants may suffice.

We repeat the above residualization step for both the variables $ X $ and $ Y $
obtaining residual matrices $ R_x $ and $ R_y $. The type of variable determines
the shape of these matrices. If the variable is continuous or ordinal, the 
residual matrix is of shape $ (n \times 1 ) $ and if the variable is categorical,
we get a residual matrix of shape $ (x \times (k - 1)) $, where $ k $ is the number
of categories of the variable.

The second step is to quantify the association between these residual matrices.
For this purpose we use canonical correlations \citep{Hotelling1936} that have been widely used to
measure the association between sets of random variables.

\begin{definition}
	Given two sets of random variables $ \bm{U} = (U_1, U_2, \cdots, U_p) $
	and $ \bm{V} = (V_1, V_2, \cdots, V_q) $, canonical correlation between
	them, $\rho_{\bm{U}, \bm{V}} $ is defined as:
		
	\begin{equation}
		% \nu_{\bm{U}, \bm{V}}= \max_{a, b} \mathrm{corr}(a^T \bm{U}, b^T \bm{V})
		\rho_{\bm{U}, \bm{V}} = \max_{a, b} \frac{a^T \Sigma_{\bm{UV}} b}{\sqrt{a^T \Sigma_{\bm{UU}} a \cdot b^T \Sigma_{\bm{VV}} b}}
	\end{equation}

	where $ a $ and $ b $ are vectors of coefficients that maximize the correlations
	between the linear combinations of $ a^T \bm{U} $ and $ b^T \bm{V} $.
\end{definition}

Canonical correlations generalize the concept of correlation coefficients to
multi-dimensional variables. It finds orthogonal linear transformations $ a $
and $ b $ that maximized the correlation between the transformed variables $
a^T \bm{U} $ and $ b^T \bm{V} $. This yields a vector of correlation
coefficient values of size $ \min(p, q) $, representing the correlation
coefficient of each pair of transformed variables. Notably, Pearson's
correlation coefficient is a special case of canonical correlations when $ p =
q = 1 $.

Several measures of association have been derived from canonical correlations, such as:
\begin{itemize}
	\item Wilks' Lambda: $\Lambda = \prod_{i}^{\min(p, q)} (1 - \rho_i^2) $
	\item Roy's Largest Root: $ \theta = \max_i(\rho_i^2) $
	\item Pillai's Trace: $ \tau = \sum_{i=1}^{\min(p, q)} \rho_i^2 $
\end{itemize}

We use a normalized version of Pillai's Trace for our purpose, given as:

\begin{equation}
	\tau_{X, Y; \bm{Z}} = \frac{1}{\min(\rvert R_x \rvert, \rvert R_y \rvert)}
	\sum_{i=1}^{\min(\rvert R_x \rvert, \rvert R_y \rvert)} (\rho_{R_x, R_y})_i^2
\end{equation}


We use Pillai's Trace because: 1) It uses all the canonical correlation values,
capturing the full extent of association, 2) Its interpretation is similar to
Pearson's correlation coefficient, i.e., $ 0 $ signifies no association and $ 1
$ signifies perfect linear relationship. We use a normalized version because
when comparing between pair of variables with different number of categories.
Our proposed measure has several desirable properties that make it well-suited
for our application:

\begin{enumerate}
	\item \textbf{Bounded: } The measure is bounded between $ 0 $ (no
		association) to $ 1 $ (perfect linear association), simplifying
		interpretation.
	% \item \textbf{Independent of Sample Size: } 
	\item \textbf{Invariant to the Dummy Encoding: } For categorical
		variables, residuals are computed using dummy encoding. Since
		canonical correlations identifies linear combinations across
		columns, this measure is invariant to the specific dummy
		encoding scheme used.
	\item \textbf{Equivalent to Partial Correlation for Continuous Variables: }
		When both $ X $ and $ Y $ are continuous variables, our measure
		is equivalent to the absolute value of partial correlation coefficient.
	\item \textbf{Equivalent to polychoric and polyserial correlation for
			ordinal and continuous variables: }
		Under the assumption that the ordinal variable is generated by
		discretizing an underlying gaussian continuous variable, this
		effect size is equivalent to polychoric and polyserial
		correlation. As both of them recover the Pearson correlation
		coefficient.
		\todo[inline]{Verify if this is correct}
	\item \textbf{Related to Cram\'er's V: } 
\end{enumerate}

In essence, our measure of conditional association extends existing metrics to mixed data, providing 
a unified measure that is interpretable.

\section{Using Measure of Association for Ranking and DAG Construction}
\label{sec:modification}

Using the mixed data measure of association presented in the last section, we
now present a method to rank modifications to a given DAG. 

Using this mixed data measure of association, we now propose a method for
iteratively constructing DAGs while incorporating expert knowledge. This
approach is based on implied CI based testing where we use the measure of
association to rank the failing CI tests. This ranking allows us to prioritize
adding edges between pair of variables that have the most severe problems.

Given a DAG $ G $ (potentially empty), a dataset $ D $, a p-value threshold $
\alpha $ and a measure of association threshold $ \beta $, our construction
procedure works as follows:

\begin{enumerate}
	\item For each pair of variables $ X $ and $ Y $ in $ G $:
		\begin{itemize}
			\item If there is an edge between $ X $ and $ Y $, remove the edge to obtain the DAG $ G' $, and compute the measure of association $ \phi(X, Y \rvert G'_{pa}(X) \cup G'_{pa}(Y)) $
			\item If there is no edge between $ X $ and $ Y $, compute the measure of association $ \phi(X, Y \rvert G_{pa}(X) \cup \underline{G}_{pa}(Y) $.
		\end{itemize}
	\item If for any pair $ X $ and $ Y $, if there is an edge between them with p-value $ > \alpha $ and $ \phi < \beta $, remove the edge between them.
	\item For the remaining $ X $ and $ Y $ pairs with p-value $ \geq \alpha $, sort them by $ \beta $. Select the pair with highest $ \beta $, ask the user to specify the direction between $ X $ and $ Y $ and add the edge to $ G $.
\end{enumerate}

For a pair of variables $ X $ and $ Y $, if there is an edge present between
them, the interpretation of this partial association measure is similar to path
coefficients in SEMs, i.e. how strong is the edge between the variables. Hence,
we can use this to identify potential edges that can be removed from the model.
In the case when there is no edge between the variables, the value of this
partial measure of association can be interpreted as - given the current
structure of the model how much of the observed correlation between the
variables $ X $ and $ Y $ in data is not explained by the model. To modify the
model based on this, we can look at the pair of variables that have an edge
between them. If any of these pair have a p-value greater than partial
association below our chosen threshold, we can remove that edge as it is
redundant.

The measure of association also gives us a way to rank the potential
modifications. The smallest values (below the threshold) of the association
measure are the worst offending ones and for when the edges are not present the
largest values would add the most amount of explainability to the model. Using
this ranking the user can decide which pair of variables to focus on.

We can then use this to rank pair of variables which have very high unexplained
association between them. Based on the domain expertise, we can then choose one
of these suggested pair of variables, decide the direction of the edge between
the variables, and add it to the model. Once the new edge is added we can
recompute the association of all other variables and $ X $ and $ Y $.

Additionally, in each iteration of modification, for each pair of variable in $
G $, that has an edge between them in we compute the partial association
measure: $ \rho(X, Y; pa_{\underline{G}}(X) \cup pa_{\underline{G}}(Y) $. This
measure of association can be interpreted as the strength of the edge between
the variables. If the value of this association below a certain threshold, we
can remove the edge.

\begin{theorem}
	If we have access to an oracle that can give correct edge directions, taking a
	greedy approach with the procedure above results in the correct graph
	being recovered. \todo[inline]{Reword and add proof}
\end{theorem}

\begin{figure}[t!]
	\begin{subfigure}{0.17 \textwidth}
		\includegraphics[page=1]{example.pdf}
		\caption{True DAG}
	\end{subfigure}%
	\begin{subfigure}{0.17 \textwidth}
		\includegraphics[page=2]{example.pdf}
		\caption{No edges}
	\end{subfigure}%
	\begin{subfigure}{0.17 \textwidth}
		\includegraphics[page=3]{example.pdf}
		\caption{Orient $ X_1 \rightarrow X_3 $}
	\end{subfigure}
	\begin{subfigure}{0.17 \textwidth}
		\includegraphics[page=4]{example.pdf}
		\caption{Orient $ X_2 \rightarrow X_3 $}
	\end{subfigure}%
	\begin{subfigure}{0.17 \textwidth}
		\includegraphics[page=5]{example.pdf}
		\caption{Orient $ X_3 \rightarrow X_4 $}
	\end{subfigure}%
	\begin{subfigure}{0.17 \textwidth}
		\includegraphics[page=6]{example.pdf}
		\caption{Orient $ X_3 \rightarrow X_5 $}
	\end{subfigure}
	\caption{An example showing a greedy approach for causal discovery using the mixed data measure of association. We start with an empty graph (b) where all the variables in the model are associated with each other. The strength of association is denoted using the thickness of the dotted edges. We select the pair of variables with the highest association, for example, $X_1$ and $X_3$, and orient it based on our domain knowledge. We continue the process till we have a DAG.
	}
\end{figure}

\subsection{Connection to Score Based Methods}

Our procedure closely resembles certain score-based automated causal discovery
methods, such as Greedy Equivalence Search (GES), where the algorithm
iteratively adds or removes edges that improve the scoring metric the most.
Similarly, a greedy version of our approach would involve adding an edge
between the pair of variables with the highest unexplained correlation. While
both methods aim to find modifications that most improve the model at each
step, a key distinction lies in the nature of the evaluation criteria: unlike
standard scoring metrics, our measure of association is not decomposable. A key
property of scoring metrics is that they should be decomposible, i.e., they can
be expressed as the sum of scores of nodes given its parents. This means adding
or removing an edge affects only a localized part of the model. In contrast,
our measure of association is global, meaning that modifying an edge in one
part of the model can influence association values elsewhere.

Another major difference is in the interpretability of the evaluation metric.
Most scoring metrics are based on log-likelihood with a penalty for model
complexity. These score metrics allow for relative comparisons between models
but do not provide an absolute measure of model fit. That is, they indicate
which model is better for a given dataset but do not reveal how well the model
explains the data in an absolute sense. In contrast, summing our measure of
association across all variables provides a direct fitness measure—this sum
approaches zero when the model perfectly accounts for all observed correlations
in the data. This property enables an absolute assessment of model quality,
rather than just a relative comparison between models.

\section{Web Tool}
\label{sec:web}
To enable users to apply this method to their own datasets, we developed an
interactive web-based tool (shown in Fig.~\ref{fig:web}) for constructing
models. Users begin by uploading their dataset, which initializes an empty DAG
with nodes corresponding to the dataset\'s variables. They can then specify a
p-value threshold and a measure of association threshold.

Using the specified thresholds, the tool visually highlights unexplained
correlations by displaying red edges between variables where correlation exists
in the data but the current DAG is not able to explain. The thickness of these
edges represents the strength of the correlation, helping users prioritize
which edges to add. Similarly, if an edge in the graph is detected to be
unnecessary, it is highlighted in black. Based on this information, users can
select to remove unnecessary edge or select a potential edge to add and specify
the orientation of the edge. The tool computes Shipley’s C \citep{Shipley2000}
value at each change to determine the overall fit of the model to the data.
Once satisfied with the constructed DAG, users can export the model for further
analysis.

\begin{figure}[t!]
	\centering
	\includegraphics[scale=0.4]{../code/plots/web_tool_full.png}
	\caption{A screenshot of the web tool for constructing the model. Users
		can upload their dataset after which the tool creates an empty
		graph and shows all pair of variables which are associated in
		the model using undirected red edges with the strength of
		association represented using edge width. Users can then
		iteratively add edges to the model (shown in green) while
		deciding the edge orientation based on domain knowledge.
		Unnecessary edges are shown in black.}
	\label{fig:web}
\end{figure}

% \begin{figure}
% 	\centering
% 	\begin{subfigure}{0.5\textwidth}
% 		\includegraphics[scale=0.25]{../../presentations/2024_05_das/2.png}
% 	\end{subfigure}%
% 	\begin{subfigure}{0.5\textwidth}
% 		\includegraphics[scale=0.25]{../../presentations/2024_05_das/5.png}
% 	\end{subfigure}
% 	\caption{Screenshots of the web-tool. \todo[inline]{Insert screenshots of the web-tool}}
% \end{figure}

\section{Empirical Analysis}
\label{sec:empirical}

\begin{figure}[t!]
	\centering
	\begin{subfigure}{0.5\textwidth}
		\centering
		\includegraphics{../code/fig3/shd_ribbon.pdf}
		\caption{}
	\end{subfigure}
	\begin{subfigure}{0.5\textwidth}
		\centering
		\includegraphics{../code/fig3/sid_ribbon.pdf}
		\caption{}
	\end{subfigure}
	\caption{Comparison of PC and Hill-Climb Search algorithms against
		manually drawn DAGs using the assistance method. As PC and
		Hill-Climb Search return the Markov Equivalence Class (MEC), we
		use the best and worst scoring orientation of the MEC to get
		the range of values. The human drawn values are done for
		multiple accuracy values of the oracle.}
\end{figure}

\begin{figure}[t!]
	\begin{subfigure}{0.25\textwidth}
		\centering
		\includegraphics{../code/fig4/unexplained_effect.pdf}
		\caption{}
	\end{subfigure}%
	\begin{subfigure}{0.25\textwidth}
		\centering
		\includegraphics{../code/fig4/ll.pdf}
		\caption{}
	\end{subfigure}
	\caption{Plots showing improvements in fit of the model over iterations
	on the Adult income dataset. For the expert simulation an LLM is used
	for the edge orientations. (a) Shows using total unexplained effect that is the
	sum of effect size between variables that don't have an edge between them. (b)
	Shows the log-likelihood of the overall model.
	}
\end{figure}

We analyzed the performance of this manual approach by comparing it with two
automated algorithms: PC and Greedy Equivalence Search (GES). For the
comparison we use simulated data from a known DAG and compare how well the
algorithm is able to recover the original model using two metrics: Structural
Hamming Distance (SHD) and Structural Intervention Distance (SID). To simulate
the data, we start with a randomly generated DAG on $ 10 $ nodes and use linear
models with random effects to generate the data. 

We vary the edge probability of the DAG to generate DAGs with varying density.
We repeat the data generation $ 30 $ times for each density value for the DAG.
As both PC and Hill-Climb Search can only give us a CPDAG, we compare the
orientation of these CPDAG which result in the best and worst case scores on
SHD and SID.

To use our proposed approach, we need an expert to tell us the direction of the
edges. We simulated an expert using this assisted model construction approach,
we start with an empty model and take a greedy approach to add edges to this
model. At any point, we select the pair of variables that has significant
correlation and the highest unexplained correlation, i.e., highest conditional
association. We then use an oracle to decide the direction of the edge between
the variables. Given an accuracy of the oracle, $ \alpha $, the oracle either
gives the correct direction, incorrect direction, or returns None representing
that it does not know the direction.

\begin{equation}
	\begin{split}
		x &= \textnormal{rand}([0, 1]) \\
		O(\alpha) &= \begin{cases} 
			M \rightarrow Y, & \textnormal{if  } x <= \alpha \\
			\textnormal{rand}(M \rightarrow Y, M \leftarrow Y, None) & \textnormal{otherwise} \\
				\end{cases} \\
	\end{split}
\end{equation}

If the oracle returns None for any pair of variables, an edge between this pair
is not suggested to the oracle in future iterations. We also make sure that
only edges which do not form a cyclic in the graph are used to select the next
potential edge. After adding each edge, we also check if the p-value of any
existing edge shows independence. If that happens we remove that edge and
blacklist that edge, i.e., that edge is not prompted again during the run of
the algorithm. We repeat this procedure till the model explains all
correlations between the variables. There is a possibility that this greedy
expert gets stuck at incorrect graph structures as it does not do any
backtracking to fix its earlier mistakes. This procedure is outlined in the
Algorithm 1.
\todo[inline]{Show an algorithm of how this expert method is working}

\begin{verbatim}
	Input: Dataset $ D $ on variables $ \bm{X} $
	Output: Learned DAG G.
	
	function compute_effects(G):
		pvalues <- dict()
		effects <- dict()
		for X_i, X_j in X:
			if edge between X_i, X_j:
				p_values[(X_i, X_j)] <- pvalue(X_i, X_j, pa_{\bar{G}}(X_i) \cup pa_{\bar{G}}(X_j))
				effects[(X_i, X_j)] <- \phi(X_i, X_j, pa_{\bar{G}}(X_i) \cup pa_{\bar{G}}(X_j))
			else:
				p_values[(X_i, X_j)] <- pvalue(X_i, X_j, pa_{\bar{G}}(X_i) \cup pa_{\bar{G}}(X_j))
				effects[(X_i, X_j)] <- \phi(X_i, X_j, pa_{\bar{G}}(X_i) \cup pa_{\bar{G}}(X_j))
		return p_values, effects
	
	$ G $ <- Empty Graph
	p_values, effects <- compute_effects(G)
	while (p_values > p_thres, 
\end{verbatim}

\begin{algorithm}
	\KwData{Data set $ D $ on variables $ \bm{X} $, pthres, effectthres}
	\KwResult{The learned DAG: $ G $}
	\BlankLine
	$ G \leftarrow $ Empty Graph
\end{algorithm}


\subsection{Using LLMs as Experts}
Recently, there has been a lot of work towards using LLMs for causal discovery,
with tasks ranging from determining pairwise edge orientation
\citep{Kiciman2023, Jin2024} to full causal structure learning \citep{Naik2023,
Vashishtha2023} and counterfactual reasoning tasks\citep{Kiciman2023} (see
\citet{Liu2024} for an overview).

Since our approach utilizes expert knowledge to determine edge orientations, we
investigated the potential of using an LLM for this task. We applied our causal
discovery procedure with using Gemini 1.5 flash model as the expert. Our method
uses a greedy approach where the pair of variables with the highest unexplained
correlation is oriented first. The prompt used to determine the orientation is
provided in Appendix~\ref{section:llms}. Our LLM prompt utilizes variable
description to determine the edge orientation. We applied this approach to the
adult income dataset and the output is shown in Fig.~\ref{fig:adult_llm}. 

Unlike the other approaches to utilize LLM for doing full causal discovery, we 
used a combination of statistical tests and LLMs. This greatly reduces the
amount of information that we need from the LLM. Our approach is also able to 
ask exact questions about the variables unlike some of the other approaches.

\begin{figure}[t!]
	\centering
	\includegraphics[page=1]{fig5.pdf}
	\caption{DAG learned using the adult income dataset using an LLM (GEMINI-1.5-flash) as the expert. The p-value threshold used is $ 0.05 $ and the measure of association threshold is $ 0. 1 $}
	\label{fig:adult_llm}
\end{figure}

\section{Conclusions}

Points:
\begin{itemize}
	\item As researchers prefer to anyways build models by hand, this is like a companion tool to help them in deciding the method.
	\item Solves all the $ 3 $ issues outlined in the introduction.
	\item Compared to other expert knowledge specification methods, users don't need to guess what information then need to provide as exact questions are asked. Also enables us to ask an LLM for it.
\end{itemize}

Future work:
\begin{itemize}
	\item Add latent variables; extend to ADMGs.
	\item Can potentially be combined with pairwise orientation methods.
	\item Improvements in the web-interface.
\end{itemize}

\bibliography{references}

\newpage

\onecolumn
\title{Expert-In-The-Loop Causal Discovery: Iterative Model Refinement Using Expert Knowledge \\ (Supplementary Material)}
\maketitle
\appendix

\section{Prompts Used for LLM}

\begin{figure}[ht!]
	\centering
	\begin{Verbatim}
You are an expert in Social Science. Following are the descriptions of two variables:

<A>: {description of variable A}
<B>: {description of variable B}

Which of the following two options is the most likely causal direction between these 
variables:

1. <A> causes <B>
2. <B> causes <A>

Return a single letter answer between the choices above; Do not provide any reasoning 
in the answer; Do not add any text formatting to the answer.
	\end{Verbatim}
	\caption{Prompt used for the LLM. Here the variable descriptions are replaced with description provided in Fig.~\ref{fig:var_description}}
	\label{fig:prompt}
\end{figure}

\begin{figure}[ht!]
	\begin{Verbatim}
          Age: The age of a person
    Workclass: The workplace where the person is employed such as Private industry, 
     	       or self employed
    Education: The highest level of education the person has finished.
MaritalStatus: The marital status of the person
   Occupation: The kind of job the person does. For example, sales, craft repair, 
   		clerical.
 Relationship: The relationship status of the person.
         Race: The ethnicity of the person.
          Sex: The sex or gender of the person.
 HoursPerWeek: The number of hours per week the person works.
NativeCountry: The native country of the person.
       Income: The income i.e. amount of money the person makes.
	\end{Verbatim}
	\caption{Variable descriptions used for prompting the LLM}
	\label{fig:var_description}
\end{figure}

\end{document}
