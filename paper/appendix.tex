\onecolumn
\title{Expert-In-The-Loop Causal Discovery: \\
Iterative Model Refinement Using Expert Knowledge \\ (Supplementary Material)}
\maketitle
\appendix

\section{Prompts Used for LLM}
\label{section:llms}

\begin{figure}[ht!]
	\centering
	\begin{verbatim}
You are an expert in Social Science. Following are the descriptions of two variables:

<A>: {description of variable A}
<B>: {description of variable B}

Which of the following two options is the most likely causal direction between these 
variables:

1. <A> causes <B>
2. <B> causes <A>

Return a single letter answer between the choices above; Do not provide any reasoning 
in the answer; Do not add any text formatting to the answer.
	\end{verbatim}
	\caption{Prompt used for the LLM. Here the variable descriptions are replaced with description provided in Fig.~\ref{fig:var_description}}
	\label{fig:prompt}
\end{figure}

\begin{figure}[ht!]
	\begin{verbatim}
          Age: The age of a person
    Workclass: The workplace where the person is employed such as Private industry, 
     	       or self employed
    Education: The highest level of education the person has finished.
MaritalStatus: The marital status of the person
   Occupation: The kind of job the person does. For example, sales, craft repair, 
   		clerical.
 Relationship: The relationship status of the person.
         Race: The ethnicity of the person.
          Sex: The sex or gender of the person.
 HoursPerWeek: The number of hours per week the person works.
NativeCountry: The native country of the person.
       Income: The income i.e. amount of money the person makes.
	\end{verbatim}
	\caption{Variable descriptions used for prompting the LLM}
	\label{fig:var_description}
\end{figure}

% \section{A Generalized Measure of Conditional Association}
% \label{sec:mixed_association}
% 
% In this section, we introduce a measure of conditional
% association for mixed data by extending the concept of partial correlation
% coefficient (commonly used for continuous variables) to mixed data. Similar to
% partial correlation coefficient, our method integrates a mixed data
% residualization method \citep{Ankan2023} with Pillai's Trace
% \citep{Pillai1955}, a multivariate measure of association based on canonical
% correlations.
%  
% Given a dataset $ D = (x, y, \bm{z}) $ on variables $ X $, $ Y $, and $ \bm{Z}
% $, our goal is to estimate the conditional association $ \phi_{X, Y; \bm{Z}} $. 
% In the first step, we compute the residuals $ R_X $ and $ R_Y $ for variables
% $ X $ and $ Y $ respectively. Depending on the type of variable the residual
% is computed as follows:
% 
% \begin{enumerate}
% 	\item \textbf{Continuous:} We train a model, $ E_X: x \sim
% 		\bm{z} $. The residuals are then computed by taking the difference
% 		between the true and the predicted values using $ E_X $. 
% 		$$ R_{x_i} = x_i - E_X(\bm{z}_i) $$
% 	\item \textbf{Ordinal:} We start by training a probability estimator, $
% 		p_X: x \sim \bm{z} $, and then use the estimated probabilities, 
% 		$ \hat{p}_X(x) $ to compute the residuals:
% 		$$ R_{x_i} = \hat{p}_X(X < x_i) - \hat{p}_X(X > x_i) $$
% 	\item \textbf{Categorical:} We again start by training a probability
% 		estimator $ p_X: x \sim \bm{z} $, and obtain probability
% 		estimates $ \hat{p}_X: p_X(\bm{z}) $. Next, we dummy encode the
% 		categorical variable, resulting in a binary vector and then
% 		compute the residuals as follows: 
% 		$$ R_{x_i} = x_i - \hat{p}_X(\bm{z}_i) $$
% \end{enumerate}
% 
% We have the option here to choose the estimators based on the characteristics
% such as distribution, type of relationship, and so on of our dataset.
% Non-parametric ensemble estimators such as Random Forest and XGBoost are robust 
% for diverse data types and complex relationships. For linear relationships,
% simpler models such as linear regression and its variants may suffice.
% 
% We repeat the above residualization step for both the variables $ X $ and $ Y $
% obtaining residual matrices $ R_x $ and $ R_y $. The type of variable determines
% the shape of these matrices. If the variable is continuous or ordinal, the 
% residual matrix is of shape $ (n \times 1 ) $ and if the variable is categorical,
% we get a residual matrix of shape $ (x \times (k - 1)) $, where $ k $ is the number
% of categories of the variable.
% 
% The second step is to quantify the association between these residual matrices.
% For this purpose we use canonical correlations \citep{Hotelling1936} that have
% been widely used to measure the association between sets of random variables.
% 
% \begin{definition}
% 	Given two sets of random variables $ \bm{U} = (U_1, U_2, \cdots, U_p) $
% 	and $ \bm{V} = (V_1, V_2, \cdots, V_q) $, canonical correlation between
% 	them, $\rho_{\bm{U}, \bm{V}} $ is defined as:
% 		
% 	\begin{equation}
% 		% \nu_{\bm{U}, \bm{V}}= \max_{a, b} \mathrm{corr}(a^T \bm{U}, b^T \bm{V})
% 		\rho_{\bm{U}, \bm{V}} = \max_{a, b} \frac{a^T \Sigma_{\bm{UV}} b}{\sqrt{a^T \Sigma_{\bm{UU}} a \cdot b^T \Sigma_{\bm{VV}} b}}
% 	\end{equation}
% 
% 	where $ a $ and $ b $ are vectors of coefficients that maximize the correlations
% 	between the linear combinations of $ a^T \bm{U} $ and $ b^T \bm{V} $.
% \end{definition}
% 
% Canonical correlations generalize the concept of correlation coefficients to
% multi-dimensional variables. It finds orthogonal linear transformations $ a $
% and $ b $ that maximized the correlation between the transformed variables $
% a^T \bm{U} $ and $ b^T \bm{V} $. This yields a vector of correlation
% coefficient values of size $ \min(p, q) $, representing the correlation
% coefficient of each pair of transformed variables. Notably, Pearson's
% correlation coefficient is a special case of canonical correlations when $ p =
% q = 1 $.
% 
% Several measures of association have been derived from canonical correlations, such as:
% \begin{itemize}
% 	\item Wilks' Lambda: $\Lambda = \prod_{i}^{\min(p, q)} (1 - \rho_i^2) $
% 	\item Roy's Largest Root: $ \theta = \max_i(\rho_i^2) $
% 	\item Pillai's Trace: $ \tau = \sum_{i=1}^{\min(p, q)} \rho_i^2 $
% \end{itemize}
% 
% We use a normalized version of Pillai's Trace for our purpose, given as:
% 
% \begin{equation}
% 	\tau_{X, Y; \bm{Z}} = \frac{1}{\min(\rvert R_x \rvert, \rvert R_y \rvert)}
% 	\sum_{i=1}^{\min(\rvert R_x \rvert, \rvert R_y \rvert)} (\rho_{R_x, R_y})_i^2
% \end{equation}
% 
% We use a normalized version of Pillai's Trace as it is bounded between $ 0 $
% and $ 1 $ and easier to interpret.
